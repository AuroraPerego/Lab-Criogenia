\documentclass[class]{article}

\begin{document}

\section{Introduzione}

\section{Cenni teorici}
\subsection{Faverzani - Pezzoli}
teoria sui segnali
\subsection{relazione}
numero di fotoni

\section{Apparato sperimentale}
\begin{itemize}
    \item criostato
    \item circuiti / IQ mixer
    \item diodo laser
    \item tutti gli strumenti usati (sint, scheda acquisizione)
\end{itemize}

\section{Funzionamento del diodo}
\begin{itemize}
    \item tipo di circuito
    \item saldature
    \item scelte di tensione e offset
    \item tipo di segnale cercato e ottenuto
\end{itemize}

\section{Conclusione}
Configurazione ottimale del diodo trovata, ora next step è implementare tutto nel criostato.
Stiamo facendo la parte di programmazione del sintetizzatore e della scheda di acquisizione.

\end{document}