\documentclass{article}

\usepackage[paper = a4paper, margin = 1in]{geometry}
\usepackage[italian]{babel}
\usepackage[utf8]{inputenc}
\usepackage{circuitikz}
\usepackage{amsmath}
\usepackage{bbold}
\usepackage{tikz}
\usepackage{amssymb}
\usepackage{caption}
\usepackage{subcaption}
\usetikzlibrary{automata, arrows.meta, positioning}

\title{Lezione teorica del laboratorio di criogenia}
\author{Aurora Perego}
\date{4 Novembre 2021}

\begin{document}

\maketitle

\section{Linea di trasmissione}
Noi facciamo l'approssimazione di componenti puntiformi per condensatori, induttori \ldots 
va bene se la lunghezza d'onda corrispondente alla frequenza a cui lavoro è più grande degli elementi. 
In questo caso si parla di \textbf{elementi concentrati} (clamped elements).
Noi invece lavoriamo a \textbf{elementi distribuiti}, cioè in una situazione in cui la lunghezza d'onda è minore delle dimensioni dei componenti.

Linea di trasmissione: necessaria per trasportare il segnale a distanze maggiori di $\lambda$, è composta di due elementi (per esempio conduttore e maglia esterna nel cavo coassiale).
In Fig.\ref{fig:lineatrasmissione} è riportato lo schema di una linea di trasmissione di lunghezza $\Delta z$.

\begin{figure}[ht]
    \centering
    \begin{circuitikz}[american voltages]
        \draw
        (0,0) to [short, *-] (7,0)
        to [C, l_=$C \Delta z$] (7,3) 
        to [short] (5,3) 
        (7,0) to [short] (9,0)
        (0,0) to [open, v^>=$\Delta V(z)$] (0,3) 
        to [short, *-,i=$i(z)$] (1,3) 
        to [R, l=$R \Delta z$] (3,3)
        to [L, l=$L \Delta z$] (6,3)
        to [R, l_=$\frac{1}{G} \Delta z$] (6,0)
        (7,3) to [short, i=$i(z+\Delta z)$] (10,3) 
        (10,0) to [open, v^>=$\Delta V(z+\Delta z)$] (10,3) 
        (7,0) to[short,-*] (10,0)
        (10,3) to[short,-*] (10,3);
        \draw[Bar-Bar] (0,-1) -- (10,-1) node[midway,sloped,above] {$\Delta z$};
    \end{circuitikz}
    \caption{Schema di una linea di trasmissione di lunghezza $\Delta z$}
    \label{fig:lineatrasmissione}
\end{figure}

Ricaviamo ora la prima equazione dei telegrafisti.
\begin{equation*}
    V(z+\Delta z,t) - V(z,t) = - R i (z,t) \Delta z - L \frac{\partial i (z,t)}{\partial t} \Delta z
\end{equation*}
Dividiamo per $\Delta z$ e facciamo il limite per $z\rightarrow 0$.
\begin{align*}
    \lim_{\Delta z \to 0} \frac{V(z+\Delta z,t) - V(z,t)}{\Delta z} &= \frac{- R i (z,t) \Delta z - L \frac{\partial i (z,t)}{\partial t} \Delta z}{\Delta z} \\
                                                                &=- R i (z,t) - L \frac{\partial i (z,t)}{\partial t}
\end{align*}
\begin{equation}\label{1eqTel}
    \frac{\partial V (z,t)}{\partial z} =  - R i (z,t) - L \frac{\partial i (z,t)}{\partial t}
\end{equation}

Partendo dalla corrente invece che dalla tensione si ricava la seconda equazione dei telegrafisti.
\begin{equation*}
    i(z+\Delta z,t) - i(z,t) = - G V (z+\Delta z,t) \Delta z - C \frac{\partial V (z+\Delta z,t)}{\partial t} \Delta z
\end{equation*}
\begin{align*}
    \lim_{\Delta z \to 0} \frac{i(z+\Delta z,t) - i(z,t)}{\Delta z} &= \frac{- G V (z+\Delta z,t) \Delta z - C \frac{\partial V (z+\Delta z,t)}{\partial t} \Delta z}{\Delta z} \\
                                                                &=- G V (z+\Delta z,t) - C \frac{\partial V (z+\Delta z,t)}{\partial t}
\end{align*}
\begin{equation}\label{2eqTel}
    \frac{\partial i (z,t)}{\partial z} = - G V (z,t) - C \frac{\partial V (z,t)}{\partial t}
\end{equation}

Nel caso ideale la resistenza della linea di trasmissione è nulla (R=0) così come la dissipazione di corrente (G=0).
In questa situazione se derivo l'eq.\ref{1eqTel} rispetto a z e l'eq.\ref{2eqTel} rispetto a t ottengo:
\begin{align*}
    \frac{\partial^2 V (z,t)}{\partial^2 z} &= - L \frac{\partial^2 i (z,t)}{\partial t \partial z}\\
    \frac{\partial^2 i (z,t)}{\partial z \partial t} &= - C \frac{\partial^2 V (z,t)}{\partial^2 t}
\end{align*}
che posso uguagliare per ottenere l'equazione delle onde per la tensione.
\begin{equation}
    \frac{\partial^2 V (z,t)}{\partial^2 z} = L C \frac{\partial^2 V (z,t)}{\partial^2 t}
\end{equation}
Questa equazione descrive un'onda progressiva ($V_p$) e una regressiva ($V_r$) che si propagano con velocità $v = \frac{1}{\sqrt{LC}}$.
Con un ragionamento analogo si ricava l'equazione d'onda per la corrente. 
Di seguito sono riportate le soluzioni delle equazioni d'onda per V e i.
\begin{align*}
    V (z,t) &= f(z-vt) + g(z+vt)\\
    i (z,t) &= \frac{1}{R_0}(f(z-vt) + g(z+vt))\\
    & \textrm{con } R_0=\sqrt{\frac{L}{C}}
\end{align*}
$R_0$ è l'impedenza caratteristica di una linea di trasmissione, indica se è presente o meno una riflessione del segnale.

Posso definire il coefficiente di trasmissione $\rho$ come:
\begin{equation*}
    \rho (z) = \frac{V_r}{V_p} = \frac{Z_L - Z_0}{Z_L + Z_0}
\end{equation*}
dove $Z_L$ è la resistenza di carico messa alla fine della linea di trasmissione.
Alcuni casi particolari:
\begin{itemize}
    \item circuito aperto ($Z_L = \infty$)$\rightarrow \rho = 1$, 
    c'è riflessione totale, l'onda torna indietro con lo stesso segno
    \item circuito chiuso ($Z_L = 0$)$\rightarrow \rho = -1$, 
    c'è riflessione totale, l'onda torna indietro con segno opposto
    \item $Z_L = Z_0 \rightarrow \rho = 0$, non c'è riflessione
\end{itemize}

Una coppia di terminali è chiamata \textbf{porta} di una rete.

\section{Matrice di scattering}
Dato un elemento con n porte da ognuna delle quali entra una tensione $V^+$ ed esce una tensione $V^-$
(schem in Fig.\ref{fig:nporte}), 
possiamo definire la matrice di scattering nel modo seguente:

\begin{equation*}
    \begin{bmatrix} 
        V^-_{1} \\
        \vdots \\
        V^-_{n}
    \end{bmatrix} 
    = 
    \begin{bmatrix} 
        S_{11} & \cdots & S_{1n} \\
        \vdots & \ddots & \\
        S_{n1} &        & S_{nn} 
    \end{bmatrix}
    \begin{bmatrix} 
        V^+_{1} \\
        \vdots \\
        V^+_{n}
    \end{bmatrix} 
\end{equation*}
\\
Gli elementi della matrice di scattering $S_{ij}=\frac{V^-_{i}}{V^+_{j}}$ rappresentano il segnale che esce da i ed entra in j (funzione di trasferimento).
Gli elementi $S_{ii}=\frac{V^-_{i}}{V^+_{i}}$ rappresentano il segnale riflesso all'ingresso dell'i-esima porta.

\begin{figure}[htbp]
    \centering   
    \begin{tikzpicture}[]
    %\filldraw[color=red!60, fill=red!5, very thick](-1,0) circle (1.5);
    
    %Nodes
    \node (mixer) [state, minimum size=2cm] {};
    \node (q2) [state, below right = of mixer] {$2$};
    \node (q1) [state, above = of mixer] {$1$};
    \node (q3) [state, below left = of mixer] {$n$};
    \node[] at (0,-2) {\textbf{$\cdots$}};
    \draw [stealth-stealth, thick]
        (mixer) edge node {}   (q1)
        (mixer) edge node {}   (q1)
        (mixer) edge node {}   (q2)
        (mixer) edge node {}   (q3);
    \draw[-stealth](0.5,1) -- (0.5,2) node[midway,sloped,right, rotate=270] {$V^-$};
    \draw[stealth-](-0.5,1) -- (-0.5,2) node[midway,sloped,left, rotate=270] {$V^+$};
    \end{tikzpicture}
    \caption{Schema di un elemento a n porte. Data la porta i il segnale in uscita è $V^-_{i}$, quello in ingresso è $V^+_{j}$}
    \label{fig:nporte}
\end{figure}

\subsection{Rete a tre porte}
Si può dimosstrare che un sistema a tre porte come quello di Fig.\ref{fig:circolatore} non può essere contemporaneamente:
\begin{itemize}
    \item lossless, cioè non dissipa potenza $\Rightarrow S S^{\dagger} = \mathbb{1}$
    \item reciproco $\Rightarrow S_{ij}=S_{ji}$
    \item senza riflessioni, cioè ha il giusto adattamento a tutti gli ingressi $\Rightarrow S_{ii}=0$
\end{itemize}

Di seguito alcuni esempi.
Il \textbf{circolatore} è uno strumento che rinuncia alla reciprocità.
La matrice di scattering è 
\begin{equation*}
    \begin{bmatrix} 
        0 & 0 & 1 \\
        1 & 0 & 0 \\
        0 & 1 & 0
    \end{bmatrix} 
\end{equation*}
per un circolatore in cui il segnale si muove in senso orario, e
\begin{equation*}
    \begin{bmatrix} 
        0 & 1 & 0 \\
        0 & 0 & 1 \\
        1 & 0 & 0 
    \end{bmatrix} 
\end{equation*}
per un circolatore in cui il segnale si muove in senso antiorario.
Un esempio di utilizzo del circolatore è come \textit{isolatore} come si può vedere in Fig.\ref{fig:isolatore}, per evitare di avere segnale riflesso dallo strumento nell'alimentazione.
L'alimentazione è collegata alla prima porta, lo strumento alla seconda e alla terza c'è un tappo da $50 \Omega$ e una messa a massa che impedisce al segnale di tornare alla prima porta. 

\begin{figure}[htbp]
    \centering
    \begin{tikzpicture}[] 
        %Nodes
        \node (mixer) [state, minimum size=2cm] {};
        \node (q2) [state, below right = of mixer] {$2$};
        \node (q1) [state, above = of mixer] {$1$};
        \node (q3) [state, below left = of mixer] {$3$};
        \draw [-, thick]
            (mixer) edge node {}   (q1)
            (mixer) edge node {}   (q1)
            (mixer) edge node {}   (q2)
            (mixer) edge node {}   (q3);
        \path [-stealth, thick,text =red]
        (0,1) edge [bend left] node [] {} (0.707,-0.707)
        (0.707,-0.707) edge [bend left] node[] {} (-0.707,-0.707)
        (-0.707,-0.707) edge [bend left] node[] {}  (0,1);
    \end{tikzpicture} 
    \caption{Schema di un circolatore} 
    \label{fig:circolatore}
\end{figure}

\begin{figure}[htbp]
    \centering
    \begin{tikzpicture}[]
        %Nodes
        \node (mixer) [state, minimum size=2cm] {};
        \node (q2) [state, rectangle, below right = of mixer] {$s$};
        \node (q1) [state, above = of mixer] {$\thicksim$};
        \node (q3) [state, below left = of mixer] {$50 \Omega$};
        \draw [-, thick]
            (mixer) edge node {}   (q1)
            (mixer) edge node {}   (q1)
            (mixer) edge node {}   (q2)
            (mixer) edge node {}   (q3);
        \path [-stealth, thick,text =red]
        (0,1) edge [bend left] node [] {} (0.707,-0.707)
        (0.707,-0.707) edge [bend left] node[] {} (-0.707,-0.707);
        \draw [-, thick]
        (-2.1,-2.5) edge node {} (-2.1,-3.5)
        (-2.6,-3) edge node {} (-1.6,-3)
        (-2.4,-3.25) edge node {} (-1.8,-3.25)
        (-2.2,-3.5) edge node {} (-2,-3.5);
    \end{tikzpicture}
    \caption{Schema di un circolatore ad utilizzo di isolatore} 
    \label{fig:isolatore}
\end{figure}


Un altro esempio è lo \textbf{splitter} che divide il segnale in due e rinucia all'essere lossless.
\begin{equation*}
    \frac{1}{2}
    \begin{bmatrix} 
        0 & 1 & 1 \\
        1 & 0 & 1 \\
        1 & 1 & 0 
    \end{bmatrix} 
\end{equation*}

Un ulteriore esempio è il \textbf{mixer} che non è reciproco e fa conversione in frequenza di segnali.
Il mixer è utile quando si usano frequenze dell'ordine del GHz che sono difficili sia da generare che da misurare.
ha due ingressi ed una uscita e può fare up-conversion o down-conversion.

\begin{figure}[htbp]
    \centering
    \begin{subfigure}{0.4\textwidth}   
        \centering
        \begin{tikzpicture}[]
            %Nodes
            \node (mixer) [state, minimum size=2cm] {};
            \node (out) [state, right = of mixer, label=below:{$f_{LO} \pm f_{IF}$}] {$f_{RF}$};
            \node (q1) [state, left = of mixer, label=below:{$f_{LO}$}] {$\thicksim$};
            \node (q2) [state, below = of mixer, label=right:{$f_{IF}$}] {$\thicksim$};
            \draw [->, thick]
                (q1) edge node {}   (mixer)
                (q2) edge node {}   (mixer)
                (mixer) edge node {}   (out);
            \draw [-, thick]
            (0.707,0.707) edge node {} (-0.707,-0.707)
            (0.707,-0.707) edge node {} (-0.707,0.707);
        \end{tikzpicture}
        \caption{Schema di un mixer in up-conversion} 
        \label{fig:upconversion}
    \end{subfigure}
    \begin{subfigure}{0.4\textwidth}
        \centering
        \begin{tikzpicture}[]
            %Nodes
            \draw [->, thick] (0,0) to (0,4);
            \draw [->, thick] (0,0) to  (6,0);
            \draw [-, thick]
            (1,0) edge node {} (1,3)
            (3,0) edge node {} (3,3)
            (4,0) edge node {} (4,3)
            (5,0) edge node {} (5,3);
            \draw []
            (1,0) edge node [label=below:{$f_{IF}$}] {} (1,0)
            (4,0) edge node [label=below:{$f_{LO}$}] {} (4,0)
            (3,3) edge node [label=above:{$f_{LO} - f_{IF}$}] {} (3,3)
            (5,3) edge node [label=above:{$f_{LO} + f_{IF}$}] {} (5,3);
        \end{tikzpicture}
        \caption{Grafico delle frequenze in ingresso e uscita di un mixer in up-conversion} 
    \end{subfigure}
    \caption{Mixer in up-conversion}
\end{figure}



\begin{figure}[htbp]
    \centering
    \begin{subfigure}{0.4\textwidth}
        \centering
        \begin{tikzpicture}[]
            %Nodes
            \node (mixer) [state, minimum size=2cm] {};
            \node (out) [state, right = of mixer, label=below:{$f_{RF} \pm f_{LO}$}] {$f_{IF}$};
            \node (q1) [state, left = of mixer, label=below:{$f_{LO}$}] {$\thicksim$};
            \node (q2) [state, below = of mixer, label=right:{$f_{RF}$}] {$\thicksim$};
            \draw [->, thick]
                (q1) edge node {}   (mixer)
                (q2) edge node {}   (mixer)
                (mixer) edge node {}   (out);
            \draw [-, thick]
            (0.707,0.707) edge node {} (-0.707,-0.707)
            (0.707,-0.707) edge node {} (-0.707,0.707);
        \end{tikzpicture}
        \caption{Schema di un mixer in down-conversion} 
        \label{fig:downconversion}
    \end{subfigure}
    \begin{subfigure}{0.4\textwidth}
        \centering
        \begin{tikzpicture}[]
            %Nodes
            \draw [->, thick] (0,0) to (0,4);
            \draw [->, thick] (0,0) to  (6,0);
            \draw [-, thick]
            (1,0) edge node {} (1,3)
            (3,0) edge node {} (3,3)
            (2,0) edge node {} (2,3)
            (5,0) edge node {} (5,3);
            \draw []
            (3,0) edge node [label=below:{$f_{RF}$}] {} (3,0)
            (2,0) edge node [label=below:{$f_{LO}$}] {} (2,0)
            (1,3) edge node [label=above:{$f_{LO} - f_{RF}$}] {} (1,3)
            (5,3) edge node [label=above:{$f_{LO} + f_{RF}$}] {} (5,3);
        \end{tikzpicture}
        \caption{Grafico delle frequenze in ingresso e uscita di un mixer in down-conversion} 
    \end{subfigure}
    \caption{Mixer in down-conversion}
\end{figure}

Nel caso dell'up-conversion ai due ingressi vengono dati una frequenza portante dell'ordine dei GHz (local oscillator, $f_{LO}$) 
e una frequenza intermedia dell'ordinde dei KHz $f_{IF}$ come si può vedere in Fig.\ref{fig:upconversion}.
In output si hanno due frequenze $f_{RF} = f_{LO} \pm f_{IF}$. 
Quello che è stato fatto è portare un segnale che era ad una bassa frequenza (banda base) ad una frequenza elevata.
Il mixer fa il prodotto tra i segnali in ingresso:
\begin{align*}
    V_{LO} = A_1 \cos(\omega_{LO}t)\\
    V_{IF} = A_2 \cos(\omega_{IF}t)\\
    V_{RF} &= k \cos(\omega_{LO}t) \cos(\omega_{IF}t) \\
           &= \frac{k}{2} (\cos((\omega_{LO}+\omega_{IF})t) + \cos((\omega_{LO}-\omega_{IF})t) )
\end{align*}

Nel caso della down-conversion invece abbasso la frequenza sottraendo dal segnale la frequenza portante.
In input ho $f_{RF}$ e $f_{LO}$, in output ottengo $f_{IF} = f_{RF} \pm f_{LO}$, come riportato in Fig.\ref{fig:downconversion}. 
La frequenza che mi interessa è la più piccola delle due e la ottengo applicando un filtro passa basso.
\begin{align*}
    V_{LO} = A_1 \cos(\omega_{LO}t)\\
    V_{RF} = A_2 \cos(\omega_{RF}t)\\
    V_{IF} &= k \cos(\omega_{LO}t) \cos(\omega_{RF}t) \\
           &= \frac{k}{2} (\cos((\omega_{RF}+\omega_{LO})t) + \cos((\omega_{RF}-\omega_{LO})t) )\\
           &\Downarrow \textrm{Filtro passa basso}\\
           &=A \cos((\omega_{RF}-\omega_{LO})t)
\end{align*}

Un altro tipo di mixer è il \textbf{mixer IQ} che è dato dall'unione di due mixer, I (segnale in fase) e Q (segnale in quadratura di fase), come riportato in Fig.\ref{fig:mixerIQ}.
In ingresso vengono date $V_{RF}$ e $V_{LO}$ ad entrambi i mixer, ma il segnale che arriva dal local oscillator prima di arrivare a Q viene shiftato di 90 gradi.
\\
Mixer I:
\begin{align*}
    V_{LO} &\propto \cos(\omega t)\\
    V_{RF} &\propto \cos(\omega t + \phi_0)\\
    V_{IF} &\propto \cos(\phi_0) + \cos(2 \omega t + \phi_0) \\
           &\Downarrow \textrm{Filtro passa basso}\\
           &\propto \cos(\phi_0)
\end{align*}
\\
Mixer Q:
\begin{align*}
    V_{LO} &\propto \cos(\omega t + \frac{\pi}{2})\\
    V_{RF} &\propto \cos(\omega t + \phi_0)\\
    V_{IF} &\propto \cos(\phi_0 - \frac{\pi}{2}) + \cos(2 \omega t + \phi_0 + \frac{\pi}{2}) \\
           &\Downarrow \textrm{Filtro passa basso}\\
           &\propto \sin(\phi_0)
\end{align*}

\begin{figure}[htbp]
    \centering
    \begin{tikzpicture}[]
        %Nodes
        \node (q1) [state] {};
        \node (q2) [state, below = of q1] {};
        \node (phase) [state] at (-1.5,-1.9) {$\phi_0$};
        \draw [-, thick]
            (q1) edge node {}   (1.5,0)
            (q2) edge node {}   (1.5,-1.9)
            (q1) edge node {}   (-1.5,0)
            (q2) edge node {}   (phase)
            (-0.3,0.3) edge node {} (0.3,-0.3)
            (0.3,0.3) edge node {} (-0.3,-0.3)
            (-0.3,-1.6) edge node {} (0.3,-2.2)
            (0.3,-1.6) edge node {} (-0.3,-2.2)
            (1.5,0) edge node {} (1.5,-1.9)
            (-1.5,0) edge node {} (phase)
            (1.5, -0.95) edge node {} (2.5, -0.95)
            (-1.5, -0.95) edge node {} (-2.5, -0.95)
            (2.5, -0.95) edge node [label=right:{$f_{RF}$}] {} (2.5, -0.95)
            (-2.5, -0.95) edge node [label=left:{$f_{LO}$}] {} (-2.5, -0.95)
            (0,0.3) edge node [label=above:{$I$}] {} (0,0.3)
            (0,-2.2) edge node [label=below:{$Q$}] {} (0,-2.2);
    \end{tikzpicture}
    \caption{Schema di un mixer IQ}
    \label{fig:mixerIQ}
\end{figure}

In questo modo ho ricavato seno e coseno di $\phi_0$ e posso fare una rappresentazione IQ del segnale.
Nel piano IQ (I sulle ascisse, Q sulle ordinate) il segnale è un vettore di ampiezza data dalla lunghezza del vettore e con fase pari all'angolo del vettore con l'orizzontale.
Un esempio è riportato in Fig.\ref{fig:rapprIQ}.

Se le frequenze LO e RF fossero diverse avrei il fenomeno dei battimenti e sul piano IQ avrei un cerchio.
\begin{figure}
    \centering
    \begin{tikzpicture}[]
        %Nodes
        \draw [->, thick] (0,0) to (0,3);
        \draw [->, thick] (0,0) to  (3,0);
        \draw [-latex] (0,0) to (2,2);
        \draw [] (1,0) to [out=90,in=315]  (0.707,0.707);
        \draw [] (0,3) edge node [label=left:{$Q$}] {} (0,3);
        \draw [] (3,0) edge node [label=below:{$I$}] {} (3,0);
        \draw [] (0.8,0.5) edge node [label=right:{$\phi_0$}] {} (0.8,0.5);
    \end{tikzpicture}
    \caption{Rappresentazione IQ del segnale}\
    \label{fig:rapprIQ}
\end{figure}

\section{Risonatori superconduttivi}
I risonatori superconduttivi possono essere realizzati con circuiti LC o con linee di trasmissione di una lunghezza tale da avere risonanza se terminata in modo opportuno.
Una linea di trasmissione cortocircuitata agli estremi dà come prima armonica un'onda di lunghezza d'onda uguale al doppio della lunghezza della linea di trasmissione, 
se fisso solo un estremo e l'altro lo lascio libero la lunghezza d'onda della prima armonica sarà 4 volte la lunghezza della linea di trasmissione.
\subsection{Rivelatori MKIDs}
Gli MKIDs sono circuiti LC con una capacità fissata dalle proprietà geometriche del circuito e una induttanza che dipende sia dalla geometria del sistema, che è fissata, che dalle proprietà del superconduttore, che variano.
Se metto una corrente continua in un superconduttore ho delle coppie di Cooper che scorrono senza resistenza.
Se invece applico una corrente alternata la situazione è diversa poiché le coppie di Cooper hanno una massa e quindi un'energia cinetica.
Quando cambia il segno della corrente cambia il trasferimento di energia cinetica, quindi cambia segno anche la velocità delle coppie, ma a causa della massa delle coppie di Cooper la forzante ha un ritardo. 
Questo fenomeno ha l'effetto di un'induttanza ed è chiamato induttanza cinetica. 
Il segnale mi dice quante coppie di Cooper ho e se mando un fotone nell'IR il numero di coppie prodotto è proporzionale all'energia del fotone.

\section{Unità di misura}
Le grandezze sono in scala logaritmica. 
Ci sono due scale: una per guadagni e attenuazioni in decibel (dB) e una per le potenze in gioco in decibel milliwatt (dBm).
Guadagno / attenuazione: 
\begin{equation}
    G = 10 \log_{10}\left(\frac{P_{out}}{P_{in}}\right) = 20 \log_{10}\left(\frac{V_{out}}{V_{in}}\right)
\end{equation}
Potenza:
\begin{align*}
    1 dBm = 10 \log_{10} P[mW]\\
    P[mW] = 10^{\frac{dBm}{10}}
\end{align*}

\section{Bibliografia}
Pozar, Microwave engineering

\end{document}
