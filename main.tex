\documentclass{article}

\usepackage[paper = a4paper, margin = 1in]{geometry}
\usepackage[italian]{babel}
\usepackage[utf8]{inputenc}
\usepackage{hyperref}
\usepackage[T1]{fontenc}
\hypersetup{
    colorlinks=true,
    linkcolor=black,
    urlcolor=blue,
}
\urlstyle{same}

% Math and physic packages
\usepackage{amsmath,amssymb,amsthm}
\usepackage{mathrsfs}
\usepackage{mathtools}
\usepackage{physics}
\numberwithin{equation}{section}

\title{Sviluppo di un sistema di controllo e di acquisizione per sorgente di singolo fotone}
\author{Marco Gobbo, Aurora Perego, Davide Vertemati}

\begin{document}

\maketitle

\section{17 Novembre 2021}
\subsection{Studio dell’apparecchio LDM9LP 
Pigtailed Laser Mount}
Modello: LDM9LP\\
\href{https://www.thorlabs.com/drawings/105dda5333f7d133-D484029A-C3D8-22D9-A82C60BDE58EE5BA/LDM9LP-Manual.pdf}{Manuale e documentazione forniti da Thorlabs}\\
%\href{Manuale.pdf}{Click here to open the file} non funziona
\href{https://www.thorlabs.com/newgrouppage9.cfm?objectgroup_id=4839&pn=LDM9LP#4839}{Informazioni generali fornite da Thorlabs}

\subsection{Calcolo rate di fotoni}
Formule utili:
\begin{equation} \label{dB}
    1 \text{ dB}_\text{m}=10 \log_{10}\left(P\left[\text{mW}\right]\right)
\end{equation}

\begin{equation} \label{energia}
    \text{Energia sorgente} [\text{J}] = P[\text{W}]\cdot t[\text{s}]
\end{equation}

\begin{equation} \label{fotoni}
    \text{Numero fotoni}=\frac{\text{Energia sorgente}}{h\nu}
\end{equation}

\noindent In elettrodinamica classica il valore di aspettazione sul vuoto di $E^2$ è $\bar{n} \frac{\hbar c}{\lambda}$, mentre in QED è $\frac{\hbar c}{\lambda^4}$.
Per questo motivo per apprezzare gli effetti quantistici dobbiamo avere che $\bar{n}\ll\frac{1}{\lambda^3}$. \\
\noindent Il laser utilizzato ha una lunghezza d'onda di $1454.4$nm ($\nu = 2.06 \cdot 10^{14}$Hz) e una potenza di $1.5$mW. Con questi dati calcoliamo il numero di fotoni emessi dal laser in un secondo tramite l'equazione \eqref{fotoni}:
\begin{equation*} 
    \text{Numero fotoni}=\frac{1.5\cdot10^{-3}\text{W}\cdot1\text{s}}{6.626 \cdot 10^{-34} \text{J s} \cdot 2.06 \cdot 10^{14}\text{Hz}} = 1.1\cdot 10^{16}
\end{equation*}
e i decibel-milliwatt dellla sorgente con l'eq. \ref{dB}:
\begin{equation*}
    1 \text{ dB}_\text{m}=10\log_{10}\left(P\left[1.5 \text{mW}\right]\right)=1.76
\end{equation*}

\noindent Per un singolo fotone a $1545$ nm:
\begin{itemize}
    \item Energia = $1.3\cdot10^{-19}$ J
    \item dB = -$158.86$
\end{itemize}

\noindent Raggiungere un singolo fotone al secondo è impossibile con gli attenuatori che abbiamo, ma soprattutto è inutile poiché considerando l'\textbf{efficienza geometrica} ne vediamo ancora meno, quindi il rate diventerebbe troppo basso. 
Facciamo il calcolo per 10 fotoni al secondo:
\begin{itemize}
    \item Energia = $1.3\cdot10^{-18}$ J
    \item dB = -$148.86$
\end{itemize}
L'attenuazione necessaria è quindi $1.76-(-148.86) = 150.62$
L'attenuazione massima a disposizione è di $156$ e corrisponde a circa $3$ fotoni.
\end{document}
