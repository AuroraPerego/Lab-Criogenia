\documentclass{article}
\usepackage[utf8]{inputenc}
\usepackage{hyperref}
\usepackage{amsmath} 

\title{Sviluppo di un sistema di controllo e di acquisitìzione per sorgente di singolo fotone}

\begin{document}

\maketitle

\section{17 Novembre 2021}
\subsection{Studio dell’apparecchio Pigtailed laser mount}
Modello: LDM9LP\\
Manuale: \url{https://www.thorlabs.com/drawings/105dda5333f7d133-D484029A-C3D8-22D9-A82C60BDE58EE5BA/LDM9LP-Manual.pdf}\\
%\href{Manuale.pdf}{Click here to open the file} non funziona
Info varie: \url{ https://www.thorlabs.com/newgrouppage9.cfm?objectgroup_id=4839&pn=LDM9LP#4839}\\

\subsection{Calcolo rate di fotoni}
Formule utili:
\begin{equation} \label{dB}
    1 dB_m=10 log_{10}\left(P\left[mW\right]\right)
\end{equation}

\begin{equation} \label{energia}
    \textrm{En. sorgente} [J] = P[W]\cdot t[s]
\end{equation}

\begin{equation} \label{fotoni}
    \# \textrm{fotoni}=\frac{\textrm{En. sorgente}}{h\nu}
\end{equation}

In elettrodinamica classica il valore di aspettazione sul vuoto di $E^2$ è $\bar{n} \frac{\hbar c}{\lambda}$, mentre in QED è $\frac{\hbar c}{\lambda^4}$.
Per questo motivo per apprezzare gli effetti quantistici devo avere che $\bar{n}<<\frac{1}{\lambda^3}$.

Il laser utilizzato ha una lunghezza d'onda di 1454.4nm ($\nu = 2.06 \cdot 10^{14}Hz$) e una potenza di 1.5mW. Con questi dati calcolo il numero di fotoni emessi dal laser in un secondo tramite l'eq. \ref{fotoni}:
\begin{equation*} 
    \# \textrm{fotoni}=\frac{1.5\cdot10^{-3}W\cdot1s}{6.626 \cdot 10^{-34}\cdot 2.06 \cdot 10^{14}Hz} = 1.1\cdot 10^{16}
\end{equation*}
e i decibel-milliwatt dellla sorgente con l'eq. \ref{dB}:
\begin{equation*}
    1dB_m=10log_{10}\left(P\left[1.5mW\right]\right)=1.76
\end{equation*}

Per un singolo fotone a 1545nm:
\begin{itemize}
    \item Energia = $1.3\cdot10^{-19}$ J
    \item dB = -158.86
\end{itemize}

Raggiungere un singolo fotone al secondo è impossibile con gli attenuatori che abbiamo, ma soprattutto è inutile poiché considerando l'efficienza geometrica ne vediamo ancora meno, quindi il rate diventerebbe troppo basso. 
Faccio il calcolo per 10 fotoni al secondo:
\begin{itemize}
    \item Energia = $1.3\cdot10^{-18}$ J
    \item dB = -148.86
\end{itemize}
L'attenuazione necessaria è quindi 1,76-(-148.86) = 150.62
L'attenuazione massima a disposizione è di 156 e corrisponde a circa 3 fotoni.
\end{document}
